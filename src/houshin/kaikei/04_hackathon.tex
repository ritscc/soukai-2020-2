\subsection*{ハッカソン方針}

%\writtenBy{\kaikeiChief}{佐田}{淳史}
\writtenBy{\kaikeiStaff}{佐田}{淳史}

\subsubsection*{目的}
ハッカソンとは,会員の技術力の向上と会員間の親睦を深めるために,合宿形式での短期開発を経験して,最終日においてその成果を発表し知識を共有する活動である.

\subsubsection*{手法}
ハッカソンは技術力の向上及び会員間の交流に役立つものであるため,夏期休暇中と冬期休暇中に 1 回行う予定である.事前にフォームで会員よりテーマのアイデアを募集した後,
参加者の募集と共に最終決定を行う.また,参加者の不足と参加申請の漏れを防ぐ目的で会員へのリマインドを積極的に行う.
グループ分けは,担当者が行い,開催1週間前までにテーマと共に参加者に通知する.
そこから開催まではアイデアソンを自由に行ってよいものとする.
これにより,ハッカソン当日は即時に開発を始めることも可能である.また,各グループ内で事前に開発に用いる言語の勉強などを行うことも可能なため,班員同士の技術的格差の解消が見込める.加えて,グループごとにリーダーを設定し,グループ内でのアイデアソンや方針の決定を主導してもらう.これにより,「会員間の親睦を深める」の目的を達成するねらいもある.その後各グループが開発を行い,最終日にその成果発表を行う.場所はエポック立命 21 で行い,期間は 3 日間とする.
発表に関しては,所要時間は 1~2 時間とし,一班あたりの上限時間は 20 分程度とする.成果物発表のためのスライド作成などは各班ごとに任意である.
参加者以外の方が見学に来た場合は特に拒まない.上回生(\fourthGrade を含む)の開発参加は許可する.学外の本会の OB などが参加したいと申し出た場合,担当者の事務処理が煩雑なものになりうるため,開催日までの時間などを鑑みて上回生会議で議論し,参加の許可を決定する.企業の参加は無しとし,協賛であれば可能とする.

\subsubsection*{次回以降の目標}
本会は,他団体と活動を掛け持ちをしている会員が多く,ハッカソンへの参加人数が少なくなる傾向がある.参加者の減少が進めば,ハッカソンの開催自体が危ぶまれる.また,一定数以上の参加がなければ,グループ数が減少し,会員間の交流など,ハッカソンにおける目標を達成しうるレベルのものを行えない可能性もある.これを受けて,指定されたテーマを受けての制作におけるアイデアと成果物の多様性を確保するため,次回以降のハッカソンでは,三つ以上のグループを用意出来るだけの人数の参加を最低目標として定める.

\subsubsection*{留意事項}
対面での実施が難しい場合,原則,オンラインでの開催を検討する.
実施方法などの詳細は,状況に応じて柔軟に決定する.
