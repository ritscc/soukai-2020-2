\subsection*{サーバ管理方針}

\writtenBy{\systemStaff}{宇佐}{基史}

サーバ管理方針として以下の五つを方針とする.
\begin{itemize}
    \item Cent Streamの調査
    \item 各種ミドルウェアの更新
    \item 立命館大学コンピュータクラブデータセンター
    \item 管理・運用の属人化を防ぐ
    \item 自主ドメインの更新など
\end{itemize}

\subsubsection*{Cent Streamの調査}
現在本会で使用しているCentOS 7のサポート期限が2024年,
後継にあたるCentOS 8のサポート期限が2021年であることを受け,
現在において対応にあたり最も有用であると考えられるCent Streamの調査を行う.
アップデート実施の是非やディストリビューションの切り替えも視野に検討していく.
必要に応じて応じて各種依存環境の調査を行い,中長期的な視点で更新を行えるように対応する.

\subsubsection*{各種ミドルウェアの更新}
本局会議内における脆弱性情報の確認,
年に複数回の最新バージョン確認を引き続き行っていく.
必要に応じてミドルウェアの更新を行う.

\subsubsection*{立命館大学コンピュータクラブデータセンター}
立命館大学コンピュータクラブデータセンター(以下,RCCDCと表記する)とは本会における情報の保存を一括して行うことを目的としたストレージの呼称である.
コロナ禍における運営の混乱の影響で実現に至らなかったRCCDCの実装を行っていく.
Windows同時ログイン時におこる不具合や操作環境からくる動作の不安定性といった,
RCCDC発案時点で存在したプロファイル問題を新たに本会で設置したNASの利用により解決を目指す.
NASの導入にあたり,サークルルーム外からのアクセスも視野にRCCDCの設計をしていく.
また,コロナ禍でも可能であれば制作も行っていく.

\subsubsection*{管理・運用の属人化を防ぐ}
サーバ管理またはサーバ運用における属人性を解決していく.
引き継ぎも適切に行い,継続して属人化を防ぐ取り組みを行っていく.

\subsubsection*{自主ドメインの更新など}
引き続き自主ドメインの更新を行う.停電対応時の作業と同時に行うことを目指す.
