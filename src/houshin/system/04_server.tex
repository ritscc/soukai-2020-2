\subsection*{サーバ管理方針}

\writtenBy{\systemStaff}{宇佐}{基史}

サーバ管理方針として以下の五つを方針とする.
\begin{itemize}
    \item Cent Streamの調査
    \item 各種ミドルウェアの更新
    \item NASの運用
    \item 管理・運用の属人化を防ぐ
    \item 自主ドメインの更新など
\end{itemize}

\subsubsection*{Cent Streamの調査}
現在本会で使用しているCentOS 7のサポート期限が2024年,
後継にあたるCentOS 8のサポート期限が2021年であることを受け,Cent Streamの調査を行う.
加えて,アップデート実施の是非やディストリビューションの切り替えも視野に検討していく.
Cent Streamの選定理由は,移行先の候補として現時点で最も有用であると考えられる為である.
必要に応じて各種依存環境の調査を行い,中長期的な視点で更新を行えるように対応する.

\subsubsection*{各種ミドルウェアの更新}
本局会議内における脆弱性情報の確認,
年に複数回の最新バージョン確認を引き続き行っていく.
必要に応じてミドルウェアの更新を行う.

\subsubsection*{NASの運用}
新しく設置されたNASの運用に伴う方針決定と実務を行う.
本会が今後行うあらゆる情報保存の統合を最終的な目的とする.
また,NASの新規導入以前にはWindowsの同時ログイン時におこる不具合や各操作環境からくる動作の不安定性といった問題が存在したが,
これ等プロファイル問題の解決も目指す.
サークルルーム外からのアクセスも視野に方針を立てる.
コロナ禍においても可能な範囲で実務を行っていく.

\subsubsection*{管理・運用の属人化を防ぐ}
サーバ管理またはサーバ運用における属人性を解決していく.
引き継ぎも適切に行い,継続して属人化を防ぐ取り組みを行っていく.

\subsubsection*{自主ドメインの更新など}
引き続き自主ドメインの更新を行う.停電対応時の作業と同時に行うことを目指す.
