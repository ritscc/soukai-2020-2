\subsection*{新歓方針}

%\writtenBy{\president}{西村}{雅貴}
%\writtenBy{\subPresident}{西村}{雅貴}
\writtenBy{\firstGrade}{西村}{雅貴}
%\writtenBy{\secondGrade}{西村}{雅貴}
%\writtenBy{\thirdGrade}{西村}{雅貴}
%\writtenBy{\fourthGrade}{西村}{雅貴}

\subsubsection*{目的}
新入生歓迎活動を行う目的として以下の2点を挙げる.
\begin{itemize}
\item 新入生に会の活動内容について知ってもらう
\item 新入生に会に興味を持ってもらう
\end{itemize}
これは本会の活動を会外に示すとともに,
新入生に会の方針を知ってもらった上で歓迎するために重要なものである.
加えて,人材確保は会の活動を維持していく上でも重要なものである.

\subsubsection*{目標}
目標に関しては,
\begin{itemize}
\item サークルルームに来てもらう
\item 企画に参加してもらう
\item 本会でやりたいことを見つけてもらう
\item 新入生の中長期的な定着
\end{itemize}
の4点を挙げ,新入生に本会に馴染んでもらえるようにする.

\subsubsection*{手法}
本項では目標を達成するために具体的な手法について述べるが,コロナ禍により状況によって変えることとする.

対面活動ができる場合,新歓期・オリエンテーション期間では,例年通りブース展示やビラ配布,サークルルームでの展示を行い,
本会がどのようなサークルであるかを分かりやすいように工夫する.
また,上回生は回生・あだ名などを記載した本会のロゴ入り名札をぶらさげ,上回生だと分かりやすくすることで,
新入生から話しかけやすい雰囲気を作る.

対面活動ができない場合,学生オフィスが主催するWeb新歓に参加し同様にサークルについての説明を行う.
また,リモートで企画を円滑に行うためにビデオ通話アプリなどを使い交流を図りやすいようにする.

\subsubsection*{企画内容}
新歓期の企画は例年同様一次企画と二次企画を行う.

新歓一次企画では,本会について,プレゼンテーション形式での総会を中心とした構成で,新入生に楽しく本会の活動を知ってもらうことを目的とする.
また,いくつかLTを行い,本会の活動内容を様々な視点から知ってもらえるように工夫する.

新歓二次企画では,コロナの状況に関わらず行えるように,オンライン実行環境などを用いて簡単なプログラミング体験ができるように資料を作成する.

新歓一次企画,二次企画以外にも,本会に馴染めるように自己紹介や交流を行う新歓交流会や,
実際に新入生に本会の活動を体験してもらうWelcomeゼミなどを実施することによって,目標を達成する.
