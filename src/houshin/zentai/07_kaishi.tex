\subsection*{会誌方針}

\writtenBy{\firstGrade}{川崎}{秀昌}

2021年度秋学期に会誌を発行する予定である.
会誌の主な目的は学園祭などで配布し,本会の活動及び結果を会外へ発信することである.
2021年度は2020年度の発行部数と同じく100部の配布を目標とする.
会誌の内容は以下の点である.
\begin{itemize}
	\item 本会紹介
	\item プロジェクト活動報告書
	\item 会員会誌
\end{itemize}
プロジェクト活動報告書及び会員コラムの執筆については,混乱を避けるために2020年度と同様のテンプレートを使用しデザインや環境の統一を行う.
また,会誌の執筆担当者は原則として\thirdGrade{}以下の会員全員であり,1人に付き1~2ページとする.
印刷形式に関しては例年通りカラー印刷とし,印刷所は2020年度と同様の会社に依頼する.
発注は余裕をもって到着するよう10月中旬に行う予定である.
