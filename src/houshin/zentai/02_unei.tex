\subsection*{運営方針}

%\writtenBy{\president}{斎藤}{竜也}
%\writtenBy{\subPresident}{斎藤}{竜也}
%\writtenBy{\firstGrade}{斎藤}{竜也}
\writtenBy{\secondGrade}{斎藤}{竜也}
%\writtenBy{\thirdGrade}{斎藤}{竜也}
%\writtenBy{\fourthGrade}{斎藤}{竜也}

2021年度の運営について,以下の4点から方針を述べる.
\begin{itemize}
    \item 定例会議
    \item 上回生会議
    \item 局
    \item 企画
\end{itemize}

\subsubsection*{定例会議}
2021年度においても,2020年度同様週1回の定例会議を行う.

定例会議では,局や企画からの連絡や会員全体ですべき議決,LTなどを行う.
また2020年度の反省を踏まえリモートでの会議がメインになった場合,定例会議後も
会員同志の交流の場として使用する.また2020年度の総括を踏まえ,
Zoomでのミーティングルーム管理を総務局が行う.

\subsubsection*{上回生会議}
2021年度においても,2020年度同様週1回の頻度にて上回生会議を行う.
執行部及び企画担当者は全員参加とする.
欠席の場合は必ず代理人を立てるようにする.
議決権のない会員に関しても2020年度同様参加の意志があればその出席を認めることとする.
局長は局会議内で上回生会議での議題を共有し,局員も議題内容を把握できるよう努める.
また,企画書の提出は会議前日までと定め,企画書のレビュー時間の短縮と円滑な議事進行に努める.

\subsubsection*{局}
2021年度は2019年度春学期のように早期局配属を実施する.
2020年度ではコロナ禍にあり,早期局配属の実施が困難であったが,各局の業務引き継ぎや会員同士の親睦を深めるなどの
観点から早期局配属を実施することが望ましい.
Welcomeゼミ終了後1か月程度で希望調査を行い,春学期総会に向けた局話し合いまでに配属を行う.
2020年度では局会議の開催頻度が低い局が見られたため,週1回の開催とし局内の一部局員のみが情報を把握している状況が起きないように努める.

\subsubsection*{企画}
本会外部との関わりがある行事に関しては\secondGrade{}の中で担当者を定め,その企画を行う.
各企画について2人以上担当者をおくようにし,2020年度の同企画担当者がそのサポートを行う.
企画書提出から企画の進捗は随時上回生会議にて報告する.また企画終了後にはKPTを上回生会議にて担当者を交えて行う.
