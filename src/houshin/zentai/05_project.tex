\subsection*{プロジェクト活動方針}

%\writtenBy{\president}{Park}{Jooinh}
%\writtenBy{\subPresident}{Park}{Jooinh}
%\writtenBy{\firstGrade}{Park}{Jooinh}
\writtenBy{\secondGrade}{Park}{Jooinh}
%\writtenBy{\thirdGrade}{Park}{Jooinh}
%\writtenBy{\fourthGrade}{Park}{Jooinh}

2020年度秋学期から継続して,プロジェクト活動を実施する.

\subsubsection*{目的と目標}
プロジェクト活動の主な目的は,情報科学の研究,およびその成果の発表を活動の基本とし
会員相互の親睦を図るとともに学術文化の創造と発展に寄与することである.

また,プロジェクト活動の目標は2020年度秋学期から継続し,以下の三つとする.

\begin{itemize}
\item 個人のみならずグループ活動としての経験を得る
\item 活動を通して技術力の向上を図る
\item 活動によって得られた成果を本会Webサイトを通して公開する
\end{itemize}

\subsubsection*{プロジェクト活動の内容}
プロジェクト活動の内容は基本的に学習または,研究要素を含むものとする.

\subsubsection*{メンバー募集}
定例会議でリーダーがプロジェクト説明をし,
その次の定例会議までに全会員がいずれかのプロジェクトに所属する.

また,個人が複数のプロジェクトに参加することも認める.

\subsubsection*{プロジェクトの設立}
プロジェクトは,以下の条件を満たした場合のみ設立できるものとする.

\begin{itemize}
\item リーダーの作成した企画書が,上回生会議で承認されること
\item メンバーが,募集終了時点でリーダーを含め3人以上であること
\item リーダーが,新入会員ではないこと
\item 1人の会員が,複数のプロジェクトのリーダを担当していないこと
\item 1人が一度に複数企画書を提出していないこと
\end{itemize}

\subsubsection*{週報}
活動後の週報の提出を義務とする.
2020年度秋学期同様に,2021年度春学期も上回生会議にて各班の週報の提出状況を確認する.

\subsubsection*{Webサイトへの公開}
プロジェクト活動が終了する前に,活動報告書を提出することを義務付ける.
報告書はPDF形式で提出し,プロジェクト発表会にてレビューを行い,
修正されたものを本会Webサイトにて公開する.
プロジェクト活動によって制作された成果物のうち,
報告書へ掲載できないものについては,
本会Webサイトでの公開を推奨するものとする.
公開を義務ではなく推奨するものとした理由は,
公開を義務付けることによるプロジェクト立ち上げ数の減少を防ぐためである.
また,公開前には不適切な内容が含まれていないかを上回生会議にて確認するものとする.

\subsubsection*{プロジェクトの運営}
週報が出ていないか週報に活動の継続が難しい旨が記述されていた場合,
プロジェクトのリーダーを上回生会議に招集し,プロジェクトの存続を問う.
その際,リーダーが招集に応じないか,またはプロジェクト存続の意思がない場合,
全ての班員を上回生会議に招集し,リーダーを受け継ぐ意志があるものが存在し,
かつ班員としてプロジェクト活動を行う意志のあるものがリーダーを含め3人以上存在する場合に限り,
プロジェクトを存続するものとする.
この時点でプロジェクトの存続が決定しなかった場合,プロジェクトは解散となる.

\subsubsection*{プロジェクトの解散}
プロジェクト配属後に班員が3人未満となるか,リーダーが欠けた場合,
そのプロジェクトは趣意書をもって理由を記述したのちに,上回生会議によって解散される.
プロジェクトが解散し,いずれのプロジェクトにも属さない会員が存在した場合,
研究推進局員が当該会員と話し合い,他のプロジェクトに配属するものとする.

\subsubsection*{プロジェクト発表会}
プロジェクト活動で得た知見や技術を共有する場としてプロジェクト発表会を行う.
プロジェクト発表会では,各班による成果発表だけでなく,報告書のレビューや質疑応答も行う.
プロジェクト発表会に向けた事前準備の時間として直前に追い込み合宿を実施する.
また,各班は2部報告書を印刷することを義務とする.
