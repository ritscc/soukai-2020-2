\subsection*{Welcomeゼミ方針}

%\writtenBy{\president}{田尻}{聖奈}
%\writtenBy{\subPresident}{田尻}{聖奈}
\writtenBy{\firstGrade}{田尻}{聖奈}
%\writtenBy{\secondGrade}{田尻}{聖奈}
%\writtenBy{\thirdGrade}{田尻}{聖奈}
%\writtenBy{\fourthGrade}{田尻}{聖奈}

Welcomeゼミを行う目的は,例年通り以下の2点である.

\begin{itemize}
    \item 新入生に本会に興味を持ってもらうこと
    \item 新入生に本会の活動を体験してもらうこと
\end{itemize}

\subsection*{目標}
\begin{itemize}
    \item 新入生に本会に慣れ親しんでもらうこと
    \item 新入生に活動に積極的に参加してもらうこと
    \item 新入生の中長期的な定着
\end{itemize}

2020年度はコロナ禍の影響を受け,オンラインでの活動が中心的であった.
2021年度も同様の状況が続く可能性があるため、,discordなどの使用を奨励していく.

\subsubsection*{手法}
Welcomeゼミの具体的な手法について,方式と内容を順に述べる.

まず,方式については例年通り,新入生の希望分野に適した上回生をあてがい開発を行う.
1人の上回生に負担が集中しないように考慮する.そのため,新入生の進捗状況は上回生全体で共有、管理する.
教導の効率化のためにも,新入生に分かりやすく教えることを念頭に置く.
エナジードリンクの類については推奨せず,エディタやプログラミング言語に関する複雑な話題は避ける.

次に,内容についても昨年同様新入生の希望を調査し,該当する分野を得意とする上回生をあてがう.
その後は新入生のペースに合わせ開発を進める.
難易度は全体的に易化し,プロダクト制作に固執せず一定の成果を上げることを目的とする.
最終日には成果発表を行う.コロナ禍の現況を鑑み,オンラインでの開催が予定されるが,状況が好転した際はその限りではない.
原則全員が行い,成果の共有と全体での発表を経験してもらう.

また,連絡手段は初回は学内メールを使用するが,以降はzoomなどのツールも臨機応変に使用する.
