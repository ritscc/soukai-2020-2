\subsection*{立命の家方針}

%\writtenBy{\president}{山本}{京介}
%\writtenBy{\subPresident}{山本}{京介}
\writtenBy{\firstGrade}{山本}{京介}
%\writtenBy{\secondGrade}{山本}{京介}
%\writtenBy{\thirdGrade}{山本}{京介}
%\writtenBy{\fourthGrade}{山本}{京介}
\subsubsection*{概要}
立命の家は毎年立命館大学で開催される小学生を対象とした企画である.
2020年度は新型コロナウイルス流行のため開催されなかった.
立命館大学に所属する学術団体やプロジェクト団体は
小学生に科学や英語,ものづくりの楽しさを知ってもらうために
様々な企画を行い,小学生と交流を深める.

\subsubsection*{目的と目標}
立命の家企画に参加する目的は,本会が学術部公認団体として
求められる還元活動の義務を果たすことである.
この目的の達成のために,本企画に参加する小学生に
本会の活動と情報技術に興味関心を向けてもらうことを目標とする.
加えて,小学生に教える体験を通して会員の教える能力の向上をはかり,
今後の還元活動を円滑にできるようになることを目標とする.

\subsubsection*{実施内容と提案}
本会ではプログラミング体験を実施する.
対象は小学1年生からで,前後半それぞれ40人を受け入れる.
教材は例年通りScratchを用い,それに加えProcessingを用いる.
Processingはビジュアルプログラミング言語であるScratchと違い,
プログラムをテキストで記述するため,Scratchより難易度が高い.
よってProcessingは例年参加してる人や高学年向けに実施する.

\subsubsection*{改善点}
2019年度の反省点を踏まえ,
\begin{itemize}
  \item 教材の準備
  \item 人員の確保
\end{itemize}
を改善点とする.

\subsubsection*{役割}
立命の家で会員に割り当てられる役職は以下の通りである.
\begin{itemize}
  \item 担当者
  \item 引率スタッフ
  \item 企画リーダー
  \item 企画スタッフ
\end{itemize} 

立命の家担当者は新\secondGrade{}に割り当てられ,立命の家実行委員として参加する.週1回の実行委員会に出席し,立命の家全体の企画と運営をする.

引率スタッフは新\firstGrade{}以上に割り当てられ,小学生を各サークルの企画会場へ引率する.
立命の家開催の数日前に行われる実行委員会主催の事前説明会に参加し,立命の家における動きや注意事項についての説明を受ける.
当日は事前説明会の内容に沿って仕事をする.

企画リーダーは新\secondGrade{}に割り当てられる.企画の立案及び主導を行い,当日は企画の進行をする.

企画スタッフは新\firstGrade{}以上に割り当てられ,企画リーダーのサポートをする.
当日は小学生たちに企画の説明をし,小学生からの質問があれば対応する.
