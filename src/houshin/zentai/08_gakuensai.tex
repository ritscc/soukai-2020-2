\subsection*{学園祭方針}

%\writtenBy{\president}{駒谷}{亮叡}
%\writtenBy{\subPresident}{駒谷}{亮叡}
\writtenBy{\firstGrade}{駒谷}{亮叡}
%\writtenBy{\secondGrade}{駒谷}{亮叡}
%\writtenBy{\thirdGrade}{駒谷}{亮叡}
%\writtenBy{\fourthGrade}{駒谷}{亮叡}

\subsubsection*{目的}
学園祭に参加する目的は以下の3点である.
\begin{itemize}
    \item 学術部公認団体としての還元活動
    \item 本会の能力向上
    \item サークル外部への情報発信
\end{itemize}
これらを目的に学園祭を運営していく.
\subsubsection*{目標}
目的を達成するために以下の3点を目標にする.
\begin{itemize}
    \item 来場者数150名
    \item 会誌頒布数100部
    \item アンケート回収率7割
\end{itemize}
また,能力向上に対する目標はプロジェクトの発表や制作物の展示を行うことで達成できると考える.
\subsubsection*{企画}
学園祭の企画として以下のものを行う.
\begin{itemize}
    \item プロジェクト活動発表
    \item LT
    \item 制作物の展示
    \item 会誌の頒布
    \item アンケート
\end{itemize}
\paragraph{プロジェクト活動発表}
2021年度も2019年度までと同様にスライドやポスターを用いてプロジェクト活動発表を行う.この発表は一般の方に向けたものであることを意識する.
ポスターについてはその場にいる会員が対応できるよう他のプロジェクトの内容を把握しておく.
2019年度までは発表機会を設けるため\firstGrade{}が担当していたが2020年度は学園祭が開催されなかったため\firstGrade{},\secondGrade{}が発表を行う.
\paragraph{LT}
2019年度までと同様に希望者を募り,制限時間5分のLTを行う.原則としてテーマは情報系に限る.また聞き手が分かりやすいように考慮したLTを心掛ける.
\paragraph{制作物の展示}
\secondGrade{},\thirdGrade{}はプロジェクト活動とは別に必ず一人一つの製作物を提出するようにする.他の回生については任意とする.
\paragraph{会誌の頒布}
2019年度までと同様に2021年度でも来場者に対して会誌の頒布を行う.
\paragraph{アンケート}
2019年度までと同様に2021年度でも反省を活かす目的でアンケートを行う.
\subsubsection*{広報物}
広報物に関しては担当の班を割り振り,円滑な広報物作成を行う.また,広報物に反映させるため早期から制作物を確定しておく.
学園祭期間には以下の手段を用いて広報を行う.
\begin{itemize}
    \item ポスター
    \item ビラ
    \item 動画
    \item 本会Webサイト
    \item 会公式Twitter
    \item 人間広告
\end{itemize}
\paragraph{ポスター}
2021年度でもテーマを早期に決定しポスターの早期完成を目指す.ポスターの内容については日時,場所,企画内容,題名を記載する.
\paragraph{ビラ}
ビラは100枚程度印刷する.またポスター同様早期完成を目指す.
\paragraph{動画}
2021年度でも動画を作成しwebサイト,会場で公開する.
\paragraph{本会Webサイト}
2019年度までと同様,トップページを学園祭仕様に置き換えることで対応する.
\paragraph{会公式Twitter}
Webサイトと同様,主に渉外局の担当とする.Twitterでは一週間前からの事前告知,開催しているLTやプロジェクト発
表などのイベント告知,展示物の紹介などを行う.
\paragraph{人間広告}
看板を作成し,学内を巡回して広告する.看板には場所,サークル名,テーマ名などを記載する.
\subsubsection*{レイアウト}
レイアウトに関しては例年,使用する部屋が決定する前にレイアウト提出を求められるため,2019年度のレイアウトを参考に提出する.部屋の決定後必要に応じてレイアウトを変更する.


