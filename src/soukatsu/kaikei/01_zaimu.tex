\subsection*{財務総括}

\writtenBy{\kaikeiChief}{小柳}{雅文}
%\writtenBy{\kaikeiStaff}{小柳}{雅文}

\subsubsection*{局会議}
決算の作成,基板助成金の申請及びそれに関連する手続きを行った.
2020年度秋学期は必要に応じてリモートで局会議を行った.
局員全員が出席できていないことが多かった.また,欠席の連絡も十分ではなかった.


\subsubsection*{会計状況の公開}
秋学期のほとんどの間公開できていなかった.上回生会議で指摘を受け,財務状況を作成,公開した.


\subsubsection*{引き継ぎ文書}
2019年度の引き継ぎ文書からの変更点を加筆,修正した.


\subsubsection*{購入申請}
新型コロナウイルスの情勢が変わらなかったため,方針の通り受け付けなかった.


\subsubsection*{会費徴収}
2020年度春学期総会の特別議案に基づき,4500円を45名から徴収した.対面での徴収が困難であったため,本会の口座への振込で徴収を行った.


\subsubsection*{2019年度学友会費について}
2019年度の学友会費のうち
\begin{itemize}
	\item[-] 4月: 31,790円
	\item[-] 5月: 52,057円
	\item[-] 中間決算: 117,007円
\end{itemize} 
が承認されていなかった.

この件について11月14日に学術本部による調査が行われ,学術本部BKC支部において会計監査業務懈怠があったことが判明した.
4月次:31,790円,中間決算:117,007円,計148,797円は補填が行われる予定である.
5月次は補正が行われなかったため,補填は行われない.


\subsubsection*{2020年度学友会費執行率について}
2020年度決算の締め切り間近まで学友会費の執行率が低いままであった.
原因は購入する物品がなかったことと,会計局長の学友会費執行率と来年度予算割当の関係の認識が不十分であったことである.


\subsubsection*{春学期基盤助成金}
NAS一式を申請していた.
採用決定されたが,学友会費の執行率が低かったため基盤助成を辞退し,申請していた商品を学友会費で購入した.


\subsubsection*{秋学期基盤助成金}
会誌の印刷費を申請した.採用内定され,現在決定申請中である.
