\subsection*{\firstGrade{}総括}

\writtenBy{\firstGrade}{川崎}{秀昌}

2020年度秋学期の\firstGrade{}総括について記述する.2020年度秋学期の\firstGrade{}方針は次の五つであった.
\begin{itemize}
	\item 親睦を深める
	\item 規律ある行動
	\item 自己発信力の向上
	\item 技術力の向上
	\item 運営に意識を向ける
\end{itemize}

\subsubsection{親睦を深める}
コロナ禍であったが,DiscordやSNSなどを通して\firstGrade{}間の交流があり,親睦を深めることができた.
しかし,上回生とはプロジェクト活動や各局会議を通して親睦を深めることができたが,個人間での親睦を深めることができなかった会員もいた.

\subsubsection*{規律ある行動}
定例会議やプロジェクト発表会では無断欠席はなかったが,事後報告が多かった.
会誌やアドベントカレンダーの締め切りを守ることができた.

\subsubsection*{自己発信力の向上}
コロナ禍であり,オンラインでモチベーションが保たれなかったことがあったため,LTをする頻度が少なかった.

\subsubsection*{技術力の向上}
プロジェクト活動やKC3,企業共同開催イベントの積極的な参加が見られ,\firstGrade{}会員の技術力が向上や,新しい知識を得ることができたと言える.
会誌やアドベントカレンダーを通して,新しい技術に挑戦した会員も見られ,締め切りまでに全会員書き終えることができた.以上と同様に技術力が向上したと言える.

\subsubsection*{運営に意識を向ける}
コロナ禍により,各局の行っていることを認識する機会が少なく,運営に意識を向けることができなかったと言える.



