\subsection*{\secondGrade{}総括}

%\writtenBy{\firstGrade}{奥川}{莞多}
\writtenBy{\secondGrade}{奥川}{莞多}
%\writtenBy{\thirdGrade}{奥川}{莞多}
%\writtenBy{\fourthGrade}{奥川}{莞多}

\secondGrade{}の2020年度秋学期活動の総括について述べる.2020年度秋学期の方針では以下の四つを方針としていた.
\begin{itemize}
    \item 期限を意識した連絡を徹底する
    \item 確実な引き継ぎを行う
    \item 下回生の模範となる活動を行う
    \item 目に見える成果を生む活動を行う
\end{itemize}

\subsubsection*{期限を意識した連絡を徹底する}
全体行事や提出物などの期限はほとんどが意識できていた.また,Slackによって再告知ができていた.

\subsubsection*{引き継ぎについて}
「それぞれの局が所管する事務を引き継ぐ」,「行事の各担当者が資料を残し,引き継ぐ」これらの2点について総括する.
まず,前者について総括する.局によっては下回生と連絡が取れておらず,完全な引き継ぎができていない.また,オンラインでは引き継ぎができないなどの理由で引き継ぎがされていない部分もあった.一方で,2019年度同様に滞りなく引き継ぎがなされている局もあった.
次に,後者について総括する.資料についてはそれぞれの行事で用意することができた.また,足りない資料があれば,担当者がフォローをすることとなった.

\subsubsection*{模範となる活動について}
遅刻・欠席の連絡については活動後の連絡が多く見られた.また,連絡なしでの遅刻・欠席もあり,連絡の少なさが目立った.

\subsubsection*{目に見える成果について}
通年で行われたプロジェクト活動については,目標をそれぞれ達成できた.
個人制作物を完成させることについては,学園祭が中止となったこともあり,目に見える成果は得られなかった.また,有志活動として発表することができなかった.
外部イベントへの参加については,オンライン活動が主な状況下であったが,勉強会などに参加している人も見受けられた.
