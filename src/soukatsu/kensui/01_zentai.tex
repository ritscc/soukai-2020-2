\subsection*{全体総括}

\writtenBy{\kensuiChief}{八木田}{裕伍}
%\writtenBy{\kensuiStaff}{八木田}{裕伍}

2020年度秋学期の研究推進局は以下の3点を目的として活動を行った.
\begin{itemize}
\item 平常活動の支援
\item 会員が興味関心のある活動ができる環境づくり
\item 発信力を養うための環境づくり
\end{itemize}

\subsubsection*{平常活動の支援}
平常活動の支援に関しては,プロジェクト活動の進捗管理やサポート,追い込み合宿,プロジェクト発表会の準備と進行を行った.
プロジェクト活動の進捗管理では,活動で生じた問題を週報で抽出し,上回生会議の議題に上げた.
2020年度秋学期は局会議を行っておらず,上回生会議での確認も怠っていたため,週報の確認ができていなかったが,
Slack のリマインダー機能を使用しプロジェクトリーダーへ週報の提出を促していたため,
ほとんどの班は週報を提出することができた.
しかし,活動を行っていなかった班を研究推進局や執行部では把握できておらず,
局会議や上回生会議での確認が必要であった.
2020年度春学期はプロジェクト活動,追い込み合宿,プロジェクト発表会はオンラインのみであり,
部屋取りは行わなかった.
プロジェクト発表会では,進行と発表の録画を行った.
予め発表時間の目安を決定していたため,長時間や短時間の発表は無く,円滑に進行できた.

\subsubsection*{会員が興味関心のある活動ができる環境づくり}
会員が興味関心のある活動ができる環境づくりは,十分行えなかったと考えられる.
原因として,毎年開催していた勉強会を,2020年度秋学期には対面での実施が難しいことから開催しなかったことが挙げられる.

\subsubsection*{発信力を養うための環境づくり}
発信力を養うための環境づくりに関しては,毎週の定例会議でLTを行った.
毎週の定例会議の時点で次週以降の担当者を通知したため,LTのための準備の時間を与えられたと考える.
2020年度秋学期には,ほぼ全てのLT担当者が担当週までにLTを行うことができた.
理由としては,局員がメールやその他Twitterなどのツールを活用して,積極的にリマインドを行ったことが挙げられる.
LT担当者は例年と比べ減少したが,これは \firstGrade の人数が少なかったこと,
\secondGrade を含めると定例会議で全ての担当者が発表できないことが理由として挙げられる.
また,飛び込みLTも例年より大幅に減少していたが,対面での課外活動を自粛していたことにより,
会員の活動が少なかったことが原因と考えられる.

また,LT意欲向上のため,LTアンケートを行い,入賞者には本会で購入する本の選択権を与えた.
2020年度秋学期は,アンケートの実施から結果の公表までを円滑に実施することが出来なかった.
これは,研究推進局が定例会議などのスケジュールを把握していなかったためである.
