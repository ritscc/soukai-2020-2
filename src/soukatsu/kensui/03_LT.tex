\subsection*{LT総括}

\writtenBy{\kensuiStaff}{深田}{紘希}

\firstGrade{}を中心にLTを行うことができた.
内容に関して,情報系に限定したが特に問題はないと判断する.
2020年度はLTのすべてをZoom上の画面録画機能を用いて記録した.
毎週メールでのリマインドを行い,ほとんどの\firstGrade{}がLTを期限内に行うことができた.

それに対して\firstGrade{}担当者のほとんどが2カ月ほどで発表を終えてしまい,LTを行う人が途切れてしまった問題がある.
しかし回生ごとに担当割り振りを行う場合\thirdGrade{}を秋学期に追加で割り振ると,一度の定例会議でLTを3人ないし4人行う必要が生じてしまい,定例会議の長期化が懸念された.
また公平性の観点から\thirdGrade{}の一部のみにLT担当者を割り振ることもできないと判断した.
よって飛び込みLTに期待して,追加の担当者割り振りは行わないことを決定した.
ところが期待に反して飛び込みLTの回数が例年を下回ったため,局長がリマインドを行い飛び込みLTの実施を促した.

LTアンケートは期限間際に実施することとなった.定例会議2週間前に準備をするためにも,局会議の実施は不可欠である.
2020年度春学期にLTアンケートを定例会議中に回答させる方式に変更すると,LTアンケートの回答率が向上したことから,これは継続して行うこととする.
2020年度はサークルルームを利用することができなかったため,LT動画に関してはサークルルーム利用が許可され次第,会内NASへアップロードする.
