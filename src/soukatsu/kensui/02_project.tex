\subsection*{プロジェクト活動総括}

%\writtenBy{\kensuiChief}{中尾}{龍矢}
%\writtenBy{\kensuiStaff}{中尾}{龍矢}

2021年度秋学期の本局プロジェクト活動方針は,会員のプロジェクト活動を円滑に進行する為,進捗確認などのサポートを行うことであった.

\subsubsection*{企画書の募集}

2020年度の方針の通り,新型コロナウイルス感染症による課外活動の制限などにより全てのプロジェクトを通年として活動したため,企画書の募集は行わなかった.

\subsubsection*{週報の回収・催促}

プロジェクトの進捗を管理する目的で各プロジェクトリーダーは週報の提出を義務付けられている.
確認された問題点は,週報の提出を遅延している班が見られた事と,催促の不足が見られた事である.
そのため,Slackのリマインダーを機能を用いて催促を行った.しかし,必要の無い場合にもリマインダーの通知が発信されていたので,機能の入切の管理する人が新たに必要とされた.

\subsubsection*{会員のプロジェクト管理}

本局では,各会員がどのプロジェクトに所属しているかを把握し,プロジェクトが途中で終了した場合などに所属していた会員のプロジェクト異動などを管理している.
2020年度春学期から引き続き,各班でリモート勉強会を開いて活動した.
一年を通して,活動に問題が見られた班があり,その原因を以下に示す.
\begin{itemize}
\item リーダーが活動を続行出来なくなった
\item 活動に使っていた資料に問題があり,継続が困難になった
\end{itemize}
一つ目の問題については,プロジェクト設立の際にリーダーの活動感想を見越した審査基準が必要と考えられた.例えば、リーダーのプロジェクト活動以外のタスク量を考慮するなど.
二つ目の問題については,週報の提出状況から班の継続不能な状況を察知し,研究推進局から事情を聴衆する必要があったと考えられた.

\subsubsection*{発表の機会の提供}
プロジェクト活動の成果発表をプロジェクト発表会を通じて行った.
プロジェクト全体を通年に変更したためか,発表を遅延する班は一つも見られなかった.
問題点を挙げるならば,レビューで指摘を行う会員が少なかったことである.