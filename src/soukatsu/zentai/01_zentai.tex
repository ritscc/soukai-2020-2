\subsection*{2020年度秋学期活動総括}

\writtenBy{\president}{原}{佑馬}

本会の目的である「情報科学の研究,及びその成果の発表を活動の基本に会員相互の親睦を図り,
学術文化の創造と発展に寄与する」ことを達成するため,方針として以下の五つを立てた.
これらについてそれぞれ評価を行うことで2020年度秋学期の総括とする.

\begin{itemize}
    \item 親睦を深める
    \item 規律ある行動
    \item 自己発信力の向上
    \item 会員間の技術向上
    \item 外部への情報発信
\end{itemize}

\subsubsection*{親睦を深める}
    2020年度秋学期活動では,主にプロジェクト活動の実施や,クリスマス会の開催を通して会員間の親睦を図った.

    2020年度秋学期のプロジェクト活動は2020年度春学期活動に引き続きオンラインで実施された.
    対面での活動が行えないことから,班員同士で気軽にコミュニケーションを取ることができるような
    機会が少なかったものの,オンラインという活動形態を鑑みれば十分親睦を深めることができたと言える.

    クリスマス会はコミュニケーションを十分取ることができるように工夫された内容であった.
    そのため,特に入会してから他の会員とコミュニケーションを取る機会が少なかった\firstGrade{}にとっては,
    参加している同回生や上回生と親睦を深める大変良い機会となった.

    また,一部の会員はZoomにおけるチャット機能や,会のDiscordサーバを活用して積極的にコミュニケーションを取っており,
    これによってより親睦を深めることができた.

\subsubsection*{規律ある行動}
    2020年度秋学期の方針として,遅刻・欠席連絡と備品整理,サークルルームの使用方法の三つの項目からなる行動規範を定めた.

    遅刻・欠席連絡は,理由が明記されていないものや,事後報告なども少なからず見られたものの,
    大半は開始時刻前に行われており,無断欠席も目立っていなかった.

    備品整理とサークルルームの使用方法については,サークルルームが原則使用不可であり,
    許可を得て立ち入った際にも,備品を持ち出すことやゴミを出すことは無かったため,
    問題は発生しなかった.

\subsubsection*{自己発信力の向上}
    自己発信力を向上させるための機会として,2020年度秋学期活動では,
    ライトニングトーク(以下,LT)やプロジェクト発表会,会誌の制作,Advent Calendarを実施した.

    定例会議におけるLTは,担当者の割り当てが無かったことから発表数自体は少なかったが,
    \firstGrade{}も意欲的に参加しており,自己発信力を向上させる良い機会となった.

    プロジェクト発表会では,定例会議と同程度の参加者が集まり,
    全てのプロジェクトの発表を滞りなく行うことができた.

    会誌とAdvent Calendarは,共に参加率が高く,内容も充実していた.
    学園祭が中止となったため,会誌の配布を行う事はできなかったが,本会Webサイト上での公開は行った.

\subsubsection*{会員間の技術向上}
    会全体の技術力を向上させることを目的として,LTやプロジェクト活動を実施した.

    定例会議におけるLTでは,\firstGrade{}と\secondGrade{}が積極的に発表しており,内容も充実していた.

    プロジェクト活動は,全ての班がオンラインで十分内容の活動を行い,
    プロジェクト発表会において,他会員に活動内容を共有することができた.

    尚,例年であれば,技術力向上の場として冬季ハッカソンを開催していたが,
    オンラインでの開催が困難であるという判断から,2020年度は開催されなかった.

\subsubsection*{外部への情報の発信}
    会外へ活動を発信する機会として,主に本会Webサイトと会公式Twitterが挙げられる.

    本会Webサイトでは,制作した会誌を掲載し,会公式Twitterでは,LTやイベントが行われる度にその様子が発信された.
    これらの頻度や内容は適切であり,本会の活動を知ってもらうためには充分であったと思われる.
