\subsection*{運営総括}

%\writtenBy{\president}{西見}{元希}
\writtenBy{\subPresident}{西見}{元希}
%\writtenBy{\firstGrade}{西見}{元希}
%\writtenBy{\secondGrade}{西見}{元希}
%\writtenBy{\thirdGrade}{西見}{元希}
%\writtenBy{\fourthGrade}{西見}{元希}

2020年度秋学期の運営について,以下の5点から総括を述べる.
\begin{itemize}
  \item 定例会議
  \item 上回生会議
  \item 局
  \item 企画
  \item 会費返金措置
\end{itemize}

\subsubsection*{定例会議}
対面での実施は春学期同様不可能であったため,
毎週木曜日にZoomを用いて実施した.
内容は変わらず局からの連絡および会員によるLTであった.
参加人数についても大きく減少するといった事例もなく,
普段通りの活動を維持できていた.

Zoomミーティングルームの準備に関しては2020年度は執行部が対応していたが,
秋学期終盤頃に総務局に担当してもらうという提案が出た.
2021年度も対面活動再開の目処がはっきりと立っていない以上,
総務局が担当する業務が少なくなっているため一考の余地があると思われる.

Slackの専用チャンネルによる議題管理は春学期同様行われなかったが,
上回生会議会議の議事録において十分な議題管理がなされていたため問題は生じなかった.

春学期同様,定例会議の運営面についてはよく活動できていたと言えるであろう.

\subsubsection*{上回生会議}
毎週水曜日に本会Discordにて実施していた.
出席に関しては特に目立った問題は無かったが,
執行部内で代理を立てない軽微な無断欠席が一部見られた.
完全リモート会議のため代理を立て辛い状況ではあるため,
ある程度は仕方ないことであったと考えられる.

その他議題関係者として企画担当者や一部プロジェクトのプロジェクトリーダーなどが招集されることがあったが,
こちらは問題なく出席していた.

議題に直接関係のない任意出席者に関しては秋学期はほとんどなく,
一部の\fourthGrade{}のみに留まっていた.
特に\firstGrade{}については任意の会員が上回生会議に出席可能であることを周知させられていなかった.

上回生会議の議題としては各局局会議内で取り扱うには大きな内容を取り上げていたが,
2020年度秋学期はそれも非常に少なく,議題が一切ない回もしばしば見られた.

2020年度の上回生会議の問題点として,議事録の情報量が少ないというものが挙げられる.
書記を明確に定めていなかったため議題担当者が話しながら記述するか気づいた出席者が記述するなどして対応していたが,
情報量や記述方法が統一されていなかったため,後から議題を参照する際に役立つ議事録となったとは言いがたい状況であった.

\subsubsection*{局}
秋学期に行った\firstGrade{}の局配属と各局の局会議について述べる.

局配属に関しては希望調査およびDiscordを用いた面談を行って可能な限り希望に沿う配属を実施した.
しかし,配属された\firstGrade{}が全員退会したり連絡が取れなくなったりしたために
他局から移籍してもらうということも起こってしまった.

局会議では,週1回の実施を確保できていない局が多く見られた.
議題があった場合のみ開催するといった状況が目立っていたが,
業務の引き継ぎや局内での情報共有,
リモート活動しか経験できていない\firstGrade{}との親交を深めるといった観点から見て,
このような状況は避けるべきであったと考えられる.
特に,局長と前局長の個人間でのみ情報共有を行っていた会計局や
そもそも局会議が一切行われていなかった研究推進局では大きな問題が生じていた.

総括すると,2020年度秋学期における局運営はリモート活動による滞りがあったと言える.

\subsubsection*{企画}
学園祭は中止となったが,
その他行事に関しては例年通りの運営を行えた.

上回生会議にて進捗状況を共有し,問題なく担当者のフォローを実施することができた.
また,企画実施後必要な行事に関してはKPTを行った.

リモートでの活動に慣れてきた秋学期は落ち着いて各行事に取り組むことができたため
円滑な企画運営ができたと言える.

\subsubsection*{会費返金措置}
2020年度春学期総会にて会費1500円の返金措置が可決されたため,
2020年度の会費を事実上減額し,4500円での請求を行った.

なお,実際の会計状況や請求方法に関しては会計局が担当していたため,
執行部としては上記の事実を述べ総括とする.
