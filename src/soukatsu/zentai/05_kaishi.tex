\subsection*{会誌総括}
\writtenBy{\secondGrade}{堀田}{隆成}

本会の活動及びその成果を会外へ発信することを目標として2020年度も会誌を作成した.
2020年度は学園祭で100冊の頒布を目標としていたが,学園祭自体が中止となったことやサークルルームが使用できなくなったことから会誌の印刷は行われておらず,外部への発信としては本会WebでのPDF公開のみとなっている.

会誌の内容には以下の3点を載せた

\begin{itemize}
    \item プロジェクト活動報告書
    \item 会員によるコラム
    \item 本会の大まかな説明・紹介
\end{itemize}

2019年度同様,会員コラム,プロジェクト活動報告書のレイアウトを統一するためGoogleスライドを用いて会員が執筆しやすい環境を整えた.
プロジェクト活動報告書に関しては活動頻度が減り報告書の内容が少なくなっているが個人コラム,本会紹介においては問題ないと考える.

内容に関しては,一般の方に向けて頒布するため,読みやすく,また読んでいて楽しいものにすることを目標としていたが,こちらは概ね達成できていた.
提出に関してはコラムは例年同様\thirdGrade{}以下の会員は原則全員提出としていたが\thirdGrade{}の提出率がよくなかった.
編集に関しては大幅に遅れが生じ,11月完成の予定がPDF版の完成が2月となった.表紙デザインに関してはデザインを担当してくれる会員を探すのに時間がかかったが,会誌全体のデザインとしては好評であった.
2020年度の会誌も著作権に注意し制作を行ったが,一部チェックが甘くなっていた.対応として問題がありそうな点が見つかった場合,執筆を担当した会員に確認を行い,実際に問題があった場合画像などを差し替えるなどして対応した.
