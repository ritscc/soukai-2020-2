\subsection*{プロジェクト活動総括}

%\writtenBy{\president}{八木田}{裕伍}
%\writtenBy{\subPresident}{八木田}{裕伍}
%\writtenBy{\firstGrade}{八木田}{裕伍}
\writtenBy{\secondGrade}{八木田}{裕伍}
%\writtenBy{\thirdGrade}{八木田}{裕伍}
%\writtenBy{\fourthGrade}{八木田}{裕伍}

\subsubsection*{全体総括}
2020年度秋学期のプロジェクト活動は,
2020年度春学期に設立された通年プロジェクトを継続して行った.
各プロジェクトには活動ごとに週報を提出することを義務付け,進捗確認を行った.
2020年度秋学期も2020年度春学期から引き続き新型コロナウイルス感染症流行のため,全ての活動をオンラインでの開催とした.

また,2020年度は通年プロジェクトのみに限定し,2020年度秋学期からの新規プロジェクトを設立しなかったため,
途中から入会した会員がプロジェクトに参加しづらいといった問題があった.

以下に2020年度秋学期に活動していたプロジェクトの一覧を示す.

\begin{itemize}
\item AlphaZero班
\item DTM班
\item Unity班
\item 競馬AI班
\item 自然言語処理班
\item C言語班
\end{itemize}

プロジェクト活動の総括は以下の六つに分けて行う.

\begin{itemize}
\item 目標の総括
\item プロジェクトの内容
\item 週報
\item 報告書
\item 追い込み合宿
\item プロジェクト発表会
\end{itemize}

\subsubsection*{目標の総括}
2020年度秋学期の目標は以下の三つであった.

\begin{itemize}
\item 活動を通して技術力の向上を図る
\item 個人のみならずグループ活動としての経験を得る
\item 活動によって得られた成果を本会Webサイトを通して公開する
\end{itemize}

これらを踏まえた総括を以下に記す.

活動を通して技術力の向上を図るに関しては,
プロジェクト活動をオンラインのみでの活動に限定していたことから,
プロジェクト進行の遅延や,
リーダーが班員の技術力を把握できなかった事態が見受けられた.
しかし,程度に差はあれど,会員のほぼ全ての技術力が向上したことから,
この目標は概ね達成できたと言える.

集団行動の重要性を学ぶに関しては,
毎回のプロジェクト活動に参加率は高く,達成できていたと言える.
しかし,途中からグループでの活動が少なく,個人活動がほとんどとなっていた班も存在した.

得られた成果を本会Webサイトを通して公開するに関しては,
プロジェクト活動報告書をWebサイトに公開されたことにより,達成できたと言える.

\subsubsection*{プロジェクトの内容}
プロジェクトの内容については,春学期から引き続き全ての班において適切であった.

\subsubsection*{週報}
2020年度秋学期は局内での週報の確認が十分でなく,
結果として途中から活動を行っていない班を研究推進局が把握できていなかった事態が発生した.
これは研究推進局で週報の確認をする局会議を行っておらず,
かつ上回生会議でも確認を行っていなかったことが原因である.

また,2019年度秋学期方針にて,週報が出ていないまたは週報に活動の継続が難しい旨が記述されていた場合,
プロジェクトのリーダーを上回生会議に招集し,
プロジェクトの存続を問うこととしており,
2020年度春学期ではUnity班が該当したが,
事態の発生が春学期の活動の終盤であったため,上回生会議に招集できなかった.
そこで2020年度秋学期ではUnity班のプロジェクトリーダーを上回生会議に招集したところ,
プロジェクトリーダーを辞退する旨を表明したため,
他の班員が引き継いだ.

\subsubsection*{報告書}

報告書の提出をもってプロジェクト終了としており,
全ての班が報告書を提出したことにより,
これを達成できた.
報告書の内容に関しては,ほぼ問題はなかった.
報告書の形式については,PDF形式で提出するものとした.

\subsubsection*{追い込み合宿}
2月4日,5日にプロジェクト発表会に向けて班員に準備を行ってもらうために,オンラインで追い込み合宿を行った.

ほぼ全ての会員が参加し,追い込み合宿で進捗を出していたことから追い込み合宿は十分な有用性があると言える.

\subsubsection*{プロジェクト発表会}

プロジェクト活動を通して得ることができた知見や技術を会内で共有する場として,オンラインでプロジェクト発表会を行った.

プロジェクト発表会は報告書をレビューを行う時間をとった後に発表,質疑応答を行うといった形式で行った.

レビューは5分間時間をとり,時間が足りなければ追加時間をとるといった形式で行った.

発表に関しては,全ての班が予め発表スライドを作成していたため,円滑に進行した.
また,発表時間を10~20分に予め制限していたため,長時間や短時間の発表は無かった.

